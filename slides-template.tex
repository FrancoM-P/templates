\documentclass{beamer}
\usepackage{amsmath,amssymb,amsthm, cancel}
\newtheorem{proposition}[theorem]{Proposition}

\title{Title}

\AtBeginSection[]{
  \begin{frame}
    \centering
    \insertsection
  \end{frame}
}

\begin{document}

\begin{frame}
\titlepage
\end{frame}

\begin{frame}
	\tableofcontents
\end{frame}

\section{Section 1}

\begin{frame}
    \frametitle{Bullet points}

    Bullet points revealing progressively:
    \begin{itemize}
        \item<2-> Bullet point 1
        \item<3-> Bullet point 2
        \item<4-> Bullet point 3 \only<5->{: an specification.}
    \end{itemize}

\end{frame}

\begin{frame}
    \frametitle{An argument}

    \only<1-6>{

    An argument layout revealing progresively:

    \pause
    \vspace{1em}
    \hspace*{2em}\begin{minipage}{0.9\textwidth}
        Premise 1

        \pause
        \vspace{1em}
        Premise 2

        \pause
        \vspace{1em}
        Premise 3

        \pause
        \vspace{1em}
        Premise 4

        \pause
        \rule{\linewidth}{0.4pt}
         Conclusion
    \end{minipage}
    }
    \only<7->{

    An argument layout revealing progresively:
    \hspace*{2em}\begin{minipage}{0.9\textwidth}
        \vspace{1em}
        \rule{\linewidth}{0.4pt}
        Conclusion

        \vspace{1em}
        Comment

    \end{minipage}
    }

\end{frame}

\begin{frame}
    \frametitle{Definition}
    
    \begin{definition}[Concept to be defined]
        A definition
    \end{definition}

    \begin{itemize}
        \item Remark 1
        \item Remark 2
    \end{itemize}

\end{frame}

\begin{frame}
    \frametitle{Callback}
    
    	Callback to an earlier slide to expand on a point 
    
        \begin{enumerate}
            \item \textcolor{gray}{This wont be discussed}
            \item \textcolor{gray}{This wont be discussed}
            \item This is what we are going to discuss next  
        \end{enumerate}
    
\end{frame}

\begin{frame}
    \frametitle{Proposition}

    \begin{proposition}[One might want to label it]
        The proposition
    \end{proposition}

\end{frame}

\begin{frame}
\frametitle{Theorem}

Introduce notation

    \begin{theorem}[Theorem's label]
        $\mbox{Some math}\Rightarrow \mbox{Rigour}$
    \end{theorem}

\end{frame}

\section{Computational setting}

\begin{frame}
\frametitle{More bullet points}

This time is something more sophisticated:

    \begin{itemize}
        \item<2-> Thing 1:
            \only<3->{Remark}
        \item<4-> Thing 2:
            \only<5->{Remark}
        \item<6-> Thing 3:
            \only<7->{Remark}         
    \end{itemize}
    
\end{frame}

\begin{frame}
\frametitle{Claim}

    \begin{block}{Label}
       Some claim.
    \end{block}

\end{frame}

\begin{frame}
    \frametitle{An image}
    
        \begin{center}
            %\includegraphics[width=0.8\textwidth]{<file-name>.<extension>}
            \footnote{Source: it might be needed to mention the source.}
        \end{center}
\end{frame}

\begin{frame}
    \frametitle{Algebraic manipulation revealing step-by-step}

    \begin{corollary}[Label]
        Let $\lambda$ be something, $c$ something else, and $p$ a third thing. Then we can make the following algebraic manipulation:
    
    % Step-by-step simplification:
    \only<1>{
    \[
    \mathbb{E}_1[C] = c (\lambda + \lambda') \frac{\lambda}{\lambda + \lambda'} p.
    \]
    }
    \only<2>{
        \[
        \mathbb{E}_1[C] = c \cancel{(\lambda + \lambda')} \frac{\lambda}{\cancel{\lambda + \lambda'}} p.
        \]
    }
    \only<3>{
        \[
        \mathbb{E}_1[C] = c \lambda p.
        \]
    }
        
    \end{corollary}
    

\end{frame}

\begin{frame}
    \frametitle{Aligned math}
        
    Some notation: Let $X_1$ conditional on $N(t) = 1$ for some $t > 0$. For any $u \in [0, t]$,
        \begin{align*}
        \Pr(X_1 \leq u \mid N(t) = 1)
        &= \frac{\Pr(X_1 \leq u,\, N(t) = 1)}{\Pr(N(t) = 1)} \\
        &= \frac{\Pr(N(u) = 1,\, N(t) - N(u) = 0)}{\Pr(N(t) = 1)} \\
        &= \frac{\lambda u e^{-\lambda u} e^{-\lambda (t - u)}}{\lambda t e^{-\lambda t}} = \frac{u}{t}.
        \end{align*}
        
        \vspace{1em}
        
        Remark.
        
\end{frame}

\begin{frame}
    \frametitle{Further structure for math}

    \only<1>{
    From our assumptions:
    \[
    \mathbb{E}[C] = c \lambda p,
    \]
    \[
    \mathbb{E}[V] = c \sum \lambda_i p.
    \]
    }
    \only<2>{
    But, what if...
    \[
    \mathbb{E}[C] = c \lambda \textbf{Cr},
    \]
    \[
    \mathbb{E}[V] = c \sum \lambda_i p,
    \]
    where $Cr$ is something.
    }
    \only<3>{
    But, what if...
    \[
    \mathbb{E}[C] = c \lambda Cr,
    \]
    \[
    \mathbb{E}[V] = c \sum \lambda_i p,
    \]
    where $Cr$ is something.
    \vspace{1em}

    Remark.
    }
    \only<4>{
    Comment.
    }

\end{frame}


\section{Appendix}

\begin{frame}
    \frametitle{More arid math stuff goes here.}
    
    A theorem
    \begin{theorem}[Label]
        The theorem
    \end{theorem}

\end{frame}


\end{document}